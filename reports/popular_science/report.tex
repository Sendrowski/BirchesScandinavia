\documentclass[10pt]{article}

\usepackage[overlay,absolute]{textpos}
\usepackage[utf8]{inputenc}
\usepackage{parskip,xcolor,titling}
\usepackage[hidelinks]{hyperref}
\usepackage[margin={10em, 10em}]{geometry}

\newcommand{\subtitle}[1]{%
  \posttitle{%
    \par\end{center}
    \begin{center}\large#1\end{center}}%
}

\begin{document}
\setlength{\droptitle}{-9em}
\title{Future of Birch Populations across Scandinavia}
\date{}
\begin{textblock}{80}(2.59,14.2)
        \begin{flushleft}
            \large Janek Sendrowski \\
            \vspace{-0.7em}
            \rule{35.37em}{1pt} \\
            \vspace{-0.02cm}
            \fontsize{9pt}{10.5pt}\color{gray}\textbf{Degree project in bioinformatics 30 hp, 2022 \\
            Biology Education Centre and Department of Ecology and Genetics, \\
            Uppsala University Supervisors: Martin Lascoux and Jennifer James \\}
        \end{flushleft}
    \end{textblock}
 
\pagenumbering{gobble}
\maketitle
\vspace{-4em}
Boreal forests are particularly vulnerable to climate change, experiencing a much more drastic increase in temperatures. The trees making up these vast and important ecosystems already had to adapt previously to environmental pressures brought about by the repeated glaciations during past ice ages. Studying the patterns of adaption of these trees can thus provide valuable insights on how to mitigate future damage. In this thesis, the structure of silver-  and downy birch populations across Scandinavia were scrutinised. The resulting understanding can lead to better management and insight on the ability of these species to adapt to future challenges. 

The silver birch, easily recognised by its eponymous silverish bark and sometimes pendulous leaves, is rather common in Europe and of great economical importance in Scandinavia, being used for timber. The downy birch, named after its downy shoots in juvenile state, is more likely encountered in the north where it gradually replaces the less cold-resistant cousin, the silver birch.

One of the most remarkable characteristics is that both species display a two-split population structure with a northern and a southern subpopulation. The boundary of these subpopulations interestingly coincides with a transitions from humid continental climate in the south to subarctic climate in the north. A similar boundary was actually found in Norway spruce in another study and it is likely that this structure is maintained because these subpopulations are better adapted to their respective climatic environments. However, the actual origin of the subpopulation structure may be due to the recolonisation of populations from different sources after the retreat of the ice sheets. The population expansion experienced after the last ice age can also leave behind a detectable genetic signature which was confirmed in this study. 

One important aspect when designing breeding strategies for silver birch for example is whether it can produce offspring with other birch species so that populations can take up genetic material from them. This uptake could aid in adapting to new challenges posed by the environment but it also complicates management considerably as it is required to keep track of several species and their relationships to get a more complete picture. One speciality of downy birches is that they possess four copies of the genetic markup instead of the more commonly found two copies present in silver birch and also in humans. This additional redundancy can provide some adaptive advantages but presents an obstacle to the interbreeding with silver birch. However, the analyses of this work approve that it is possible for some genetic material of silver birch to make its way into downy birch but not vice versa.

The origins of the genetic architecture of downy birch are difficult to trace back but at some point in the past, the genetic material of one species was duplicated or that of two different species fused to give rise to its ancestor. Some evidence indicates that the duplication of a species very similar to today's silver birch gave rise to downy birch. This hypothesis is strengthened by an analysis which reveals that the impact new mutations have on their carriers' fitness is remarkably similar between the two species. Another important aspect is to find out whether our species harbour a large amount of disadvantageous mutations. This can be a sign that the population is in ill heath and in decline. As it turns out, there is no cause of concern with the amount of such mutations being low in comparison to other tree species like conifers.

To summarise, this work confirmed the existence of two genetic clusters in each species, the unidirectional transfer of genetic material from silver birch into downy birch, and population expansion after the last ice age. The distribution of the effect size of new mutations between the two species is also surprisingly similar and the amount of disadvantageous mutations seems acceptable. There is thus hope that these species will be able to adapt to future environmental challenges and that the silver birch continues being a good and sustainable resource for timber. Much remains to be done to unravel the complex interplay and history of these rather flexible species whose prosperity is crucial to a functioning boreal ecosystem.


\end{document}
